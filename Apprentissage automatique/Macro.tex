%-----------------------------------------------------------%
%-----------------------------------------------------------%
%-------------- Les biblioth�ques MikTex -------------------%
%-------------- pour la r�dation d'un  ---------------------%
%-------------- document scientifique  ---------------------%

\usepackage[french]{babel}
\usepackage[ansinew]{inputenc}    %Elle prend en charge les caract�res accentu�s
\usepackage{amsfonts}
\usepackage{amssymb}
\usepackage{amsmath}
\usepackage{geometry}
\usepackage{graphics}
\usepackage{graphicx}
%\usepackage{a4wide}
%\usepackage[chapter]{algorithm}
%\usepackage{algorithmic}
\usepackage[lofdepth,lotdepth]{subfig}
%\usepackage{subfigure}
\usepackage{theorem}
\usepackage{amstext}
\usepackage{newlfont}
\usepackage{epsfig}
\usepackage{float}
\usepackage{xspace}
\usepackage{here}
\usepackage{caption}
\usepackage{pgfplots}
\usepackage{fancyhdr}
\usetikzlibrary{patterns}
\captionsetup[subfigure]{subrefformat=simple,labelformat=simple}
\renewcommand\thesubfigure{(\alph{subfigure})}
%\usepackage{systeme}
\usepackage{pdfpages}  %Permet d�inclure dans votre document des pages compl�tes d�un autre document PDF
\usepackage{color}
\usepackage[colorlinks,linkcolor=blue]{hyperref}
%\usepackage[12pt]{extsizes} %La taille de la police:  8pt, 9pt, 10pt, 11pt, 12pt, 14pt, 17pt et 20pt.
%%------------------------------------------------------------%%
%%------------------------------------------------------------%%

\inputencoding{ansinew}   % elle suit le package[ansinew]{inputenc}.

%------------------------------------------------------------%
%---------Personnalisation des titres des chapitres----------%

%\usepackage[Conny]{fncychap}      %Style1:% Necessite d'installer le paquet fncychap de www.ctan.org/pkg/fncychap
\usepackage[Lenny]{fncychap}      %Style2:% Necessite d'installer le paquet fncychap de www.ctan.org/pkg/fncychap
%\usepackage[Sonny]{fncychap}      %Style3:% Necessite d'installer le paquet fncychap de www.ctan.org/pkg/fncychap
%\usepackage[Rejne]{fncychap}      %Style5:% Necessite d'installer le paquet fncychap de www.ctan.org/pkg/fncychap
%\usepackage[Bjarne]{fncychap}      %Style6:% Necessite d'installer le paquet fncychap de www.ctan.org/pkg/fncychap
%\usepackage[Bjornstrup]{fncychap}  %Style7:% Necessite d'installer le paquet fncychap de www.ctan.org/pkg/fncychap

%------------------------------------------------------------%
%--------------Mise en page du document----------------------%

\geometry{left=2cm,right=2cm,top=2cm,bottom=2cm} %D�finir les marges des pages du documment.
\renewcommand\headrulewidth{1pt}

%%-------------------------------------------------
\lhead{\slshape \nouppercase{\bfseries\leftmark}}
\rhead{\slshape \nouppercase{\bfseries\rightmark}}
\cfoot{\bfseries\thepage}
%------------------------------------------------------------%
%--Abr�viation des caract�res des ensembles math�matique-----%
\newcommand{\Z}{\mathbb Z}
\newcommand{\R}{\mathbb R}
\newcommand{\N}{\mathbb N}
\newcommand{\C}{\mathbb C}
\newcommand{\Q}{\mathbb Q}
%-------------------------------------------------------------%
%----------D�finir les environements du document--------------%


\newcounter{theorem}                       %Pour initialiser la num�rotation des environement � partir de 1.
\theoremstyle{definition}                  %D�finit le style du theoreme M�me chose que \theoremstyle{remark}.
%\theoremstyle{example}                    %M�me chose que \theoremstyle{plain} marche avec le package amsthm.
\newtheorem{theoreme}{Th�or�me}[section]   %La numeration des environements theoreme est d�clar�e par section(aussi [chapter];[part];[susection])
\newtheorem{proposition}[theoreme]{Proposition}
\newtheorem{corollary}[theoreme]{Corollaire}
\newtheorem{lemma}[theoreme]{Lemme}
\newtheorem{definition}[theoreme]{D�finition}
\newtheorem{remark}[theoreme]{Remarque}
\newtheorem{theorem}[theoreme]{Th�or�me}
\newtheorem{example}[theoreme]{Exemple}



